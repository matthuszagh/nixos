\documentclass{default}

\begin{document}

\tableofcontents
\hypersetup{linkcolor=red}

\chapter{Installation}\label{cha:installation}

I've fetched Emacs from the \href{https://github.com/emacs-mirror/emacs}{GitHub mirror
  repository}. The source code resides in \textasciitilde/developer/software/emacs. Installation
instructions reside in the source directory in the file named ``INSTALL''. I've chosen to use a
build directory named ``build''. From the build directory run configure with:

\begin{minted}{bash}
$ ./../configure --with-x-toolkit=gtk3 --with-mailutils \
                 --with-imagemagick --with-xwidgets
\end{minted}

I'm currently using version 26.1, by running:

\begin{minted}{bash}
$ git checkout emacs-26.1
\end{minted}

There is some issue with highlighting in version 27.050 (potentially font-lock?) that makes it very
annoying to write code. In general I should only use stable releases, which are specified on the
\href{https://www.gnu.org/software/emacs/}{Emacs website}.

\chapter{Packages}\label{cha:packages}

\section{Gnus}
\label{sec:gnus}

Gnus was a huge pain to setup. I finally got it working with the information at this
\href{https://eschulte.github.io/emacs-starter-kit/starter-kit-gnus-imap.html}{link}.

\section{Term}\label{sec:term}

Ansi term uses 8 colors for the terminal GUI. We can redefine these colors in an Emacs terminal by
customizing their values with \mintinline{text}{M-x customize-group RET term RET}. The colors must
also be configured in \mintinline{text}{~/.bashrc}.

\chapter{Structure}\label{cha:structure}


\chapter{Best Practices}
\label{cha:best-practices}

\href{https://www.gnu.org/software/emacs/manual/html_node/elisp/Key-Binding-Conventions.html}{This
  Emacs manual page} contains principles to follow when customizing key bindings. Basically,
\mintinline{text}{C-c <letter>} (but not \mintinline{text}{C-c} followed by another control
character) as well as \mintinline{text}{<f5>} through \mintinline{text}{<f9>} are free for users to
define how they wish.


\chapter{To-Do}\label{cha:to-do}

\section{Ivy/Swiper to Helm}\label{sec:ivyswiper-helm}

Consider changing ivy and swiper to Helm. Helm is a larger project and may therefore be better
maintained with more features. The downside to this might be slower autocompletion times in which
case it may not be worth changing. If you do make the change, make sure to adjust
\lstinline{custom-set-faces} for Helm instead of ivy.


\section{Powerline Custom Mode-Line Configuration for Specific Modes}
\label{sec:powerl-cust-mode}

I'm experiencing an issue with powerline. I'd like to be able to display 2 different mode lines
depending on whether I'm in pdf-view-mode or not. When displaying a PDF I'd like to display the
current page and the total number of pages instead of the current row/column position of the
cursor. In all other modes I'd like to display the row/column position of the cursor. The issue
seems to be that powerline does not allow setting different configurations for different major
modes. For instance, if I set a hook for pdf-view-mode, it will then stay in force for other
buffers. I'm not eager to go back to the default mode line because it doesn't show the file size and

powerline developers that different modes are not possible. Another possible solution might be
spaceline. Or, maybe I can get the buffer size in the default mode line display (look at powerline
function).

The file size can be displayed in the mode line of certain buffers by enabling
\mintinline{text}{size-indication-mode}. Critically though, this doesn't work when displaying PDF's
and potentially other file types as well. I've currently disabled powerline until I have a better
solution: seeing the PDF page number outweighs the cost of a slightly worse aesthetic and seeing the
file size for PDFs.

\section{Disable cursor in pdf-view-mode}

I'd like to disable the cursor when viewing a PDF. It's distracting and provides no value. I've
tried these two additions to \mintinline{text}{init.el} to no avail:

\begin{minted}{elisp}
(add-hook 'pdf-view-mode-hook
          (lambda ()
            (make-variable-buffer-local 'cursor-type)
            (setq cursor-type nil)))
(add-hook 'post-command-hook
          (lambda ()
            (setq cursor-type (if pdf-view-mode t 'nil))))
\end{minted}

These are the links I've found that address related issues:

\href{https://emacs.stackexchange.com/questions/392/how-to-change-the-cursor-type-and-color}{How to
  change the cursor type and color?}

\href{https://www.gnu.org/software/emacs/manual/html_node/elisp/Cursor-Parameters.html}{Cursor
  parameters}

\href{https://emacs.stackexchange.com/questions/44650/how-can-i-make-the-cursor-change-to-block-in-overwrite-mode?rq=1}{How
  can I make the cursor change to block in overwrite mode?}

\href{https://www.emacswiki.org/emacs/ChangingCursorDynamically}{Changing Cursor Dynamically}

\href{https://www.gnu.org/software/emacs/manual/html_node/emacs/Cursor-Display.html}{Cursor Display}


\section{Run Emacs as Daemon}

Apparently, running Emacs as a server and then using emacsclient gives better results. I should
investigate how this is done.


\section{Get much better at Gnus}



\section{Display header when buffer is displayed}

Currently, a header displays only when you cursor to it. It should be displayed immediately when the
buffer opens.


\section{Remap keybindings to adhere to best practices}



\end{document}