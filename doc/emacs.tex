\documentclass{default}

\begin{document}

\tableofcontents
\hypersetup{linkcolor=red}

\chapter{Installation}\label{cha:installation}

I've fetched Emacs from the \href{https://github.com/emacs-mirror/emacs}{GitHub mirror
  repository}. The source code resides in \textasciitilde/developer/software/emacs. Installation
instructions reside in the source directory in the file named ``INSTALL''. I've chosen to use a
build directory named ``build''. From the build directory run configure with:

\begin{minted}{bash}
$ ./../configure --with-x-toolkit=gtk3 --with-mailutils \
                 --with-imagemagick --with-xwidgets
\end{minted}

I'm currently using version 26.1, by running:

\begin{minted}{bash}
$ git checkout emacs-26.1
\end{minted}

There is some issue with highlighting in version 27.050 (potentially font-lock?) that makes it very
annoying to write code. In general I should only use stable releases, which are specified on the
\href{https://www.gnu.org/software/emacs/}{Emacs website}.

\chapter{Package Setup}\label{cha:package-setup}

\section{Gnus}
\label{sec:gnus}

Gnus was a huge pain to setup. I finally got it working with the information at this
\href{https://eschulte.github.io/emacs-starter-kit/starter-kit-gnus-imap.html}{link}.

\section{Term}\label{sec:term}

Ansi term uses 8 colors for the terminal GUI. We can redefine these colors in an Emacs terminal by
customizing their values with \mintinline{text}{M-x customize-group RET term RET}. The colors must
also be configured in \mintinline{text}{~/.bashrc}.

\chapter{Functionality}
\label{cha:functionality}

\section{Code Tagging and Navigation}
\label{sec:code-tagging}


\chapter{Structure}\label{cha:structure}


\chapter{Best Practices}
\label{cha:best-practices}

\href{https://www.gnu.org/software/emacs/manual/html_node/elisp/Key-Binding-Conventions.html}{This
  Emacs manual page} contains principles to follow when customizing key bindings. Basically,
\mintinline{text}{C-c <letter>} (but not \mintinline{text}{C-c} followed by another control
character) as well as \mintinline{text}{<f5>} through \mintinline{text}{<f9>} are free for users to
define how they wish.


\chapter{To-Do}\label{cha:to-do}

\section{Customize Mode Line}
\label{sec:customize-mode-line}

I can customize mode-line-format on a per-mode basis (see
\href{https://emacs.stackexchange.com/questions/13652/how-to-customize-mode-line-format}{this
  StackExchange question}). Also, see the
\href{https://www.gnu.org/software/emacs/manual/html_node/elisp/Mode-Line-Variables.html#Mode-Line-Variables}{Emacs
  documentation} on the subject.

\section{Disable cursor in pdf-view-mode}

I'd like to disable the cursor when viewing a PDF. It's distracting and provides no value. I've
tried these two additions to \mintinline{text}{init.el} to no avail:

\begin{minted}{elisp}
(add-hook 'pdf-view-mode-hook
          (lambda ()
            (make-variable-buffer-local 'cursor-type)
            (setq cursor-type nil)))
(add-hook 'post-command-hook
          (lambda ()
            (setq cursor-type (if pdf-view-mode t 'nil))))
\end{minted}

These are the links I've found that address related issues:

\href{https://emacs.stackexchange.com/questions/392/how-to-change-the-cursor-type-and-color}{How to
  change the cursor type and color?}

\href{https://www.gnu.org/software/emacs/manual/html_node/elisp/Cursor-Parameters.html}{Cursor
  parameters}

\href{https://emacs.stackexchange.com/questions/44650/how-can-i-make-the-cursor-change-to-block-in-overwrite-mode?rq=1}{How
  can I make the cursor change to block in overwrite mode?}

\href{https://www.emacswiki.org/emacs/ChangingCursorDynamically}{Changing Cursor Dynamically}

\href{https://www.gnu.org/software/emacs/manual/html_node/emacs/Cursor-Display.html}{Cursor Display}

Instead of trying to get rid of the cursor, maybe just make its color the same as the
background. Check out these links:
\href{https://emacs.stackexchange.com/questions/7281/how-to-modify-face-for-a-specific-buffer}{se}
and \href{https://www.gnu.org/software/emacs/manual/html_node/elisp/Face-Remapping.html}{manual}.

\section{Get much better at Gnus}

\section{Flycheck doesn't work with C}

I'm getting an error that the flag -std=c++17 doesn't work with
C\@. \href{https://github.com/alexmurray/flycheck-clang-analyzer/issues/6}{This error seems somewhat
similar}.

\section{load-file from Emacs}

This causes cursor color to change and potentially other behavior as well.

\section{align-current AUCTeX}

align-current works well for tables with limited text. However, it propagates line breaks and so
takes up a needless number of lines for long text sections that have to be manually adjusted. Also,
I'd like to be able to run this globally for a whole buffer. The most similar action is align-entire
but this aligns all tables to one another; I'd like aligning to be done on a per-table basis.

\section{Consider switching to sane-term from multi-term}

\href{https://github.com/adamrt/sane-term}{sane-term} seems like it might be a better alternative to multi-term.

\section{Single quote delimeter not working}

Pressing \' twice causes three single quotes to be inserted.

\section{Command that closes all buffers for which underlying file was deleted}

Probably better, just do this automatically. One worry is if file was accidentally deleted and
buffer allows you to keep the file alive. Still, existing buffers shouldn't be used as a sort of
backup.

\section{Magit bug missing headers}

I'm getting the error: ``BUG: missing headers nil'' when using magit with ghub. This is a known
issue and is tracked \href{https://github.com/magit/ghub/issues/81}{here}. I guess the solution is
to just wait for someone to fix it there.

\section{Verilog mode escape}

When in evil insert mode and the cursor position is after a reg or wire, pressing escape tries to
vectorize the reg/wire instead of entering evil normal state.

\section{Verilog code tagging}

Verilog mode should have code tagging and navigation. See
\href{https://scripter.co/ctags-systemverilog-and-emacs/}{this link} for ideas on code tagging.

\section{Better Code Tagging}

The problem with RTags is that it only works for C-based languages. gtags should work for many
others. See \href{https://tuhdo.github.io/c-ide.html}{this link}. It probably makes sense to use the
\href{https://github.com/leoliu/ggtags}{ggtags} interface. Also use the
\href{https://github.com/syohex/emacs-helm-gtags/}{helm interface}. This may also be the wrong
question, but maybe it's also possible to use rtags as a backend for a common gtags frontend?

\section{Make Comint mode behave more like term mode}

It's annoying that term mode and comint mode use different bindings for navigating command history
(C-p and M-p, respectively). They should both use C-p. Additionally, find a way to use the
equivalent of term-line-mode/term-char-mode for comint.

\section{Create Add-On that allows you to edit Anki HTML in Emacs}

Adapt \href{https://github.com/louietan/anki-editor}{this} for your needs. That add-on should
contain nearly all of the required functionality.

\section{Get XWidget Webkit working}

See \href{https://emacsnotes.wordpress.com/2018/08/18/why-a-minimal-browser-when-there-is-a-full-featured-one-introducingxwidget-webkit-a-state-of-the-art-browser-for-your-modern-emacs/}{this}.

\section{Get LaTeX SVGs working in EWW}

EWW is unable to render latex/mathjax directly and so renders the resulting svg instead. This
appears as dark text on a dark background. See
\href{https://emacs.stackexchange.com/questions/3622/use-a-different-color-theme-for-eww-buffers}{this
  post}.

\section{Improve org-books}

\begin{enumerate}
\item Query user for file when running org-capture and base title on that.
\item Make table of contents entries links to the corresponding page in the pdf.
\item Write a program to extract a table of contents from pdf. Run this automatically when running
  org-capture.
\item Bind org-capture to a keybinding.
\item Write scripts to get info from Goodreads and Amazon (Goodreads API
  \href{https://www.goodreads.com/api}{here}).
\item Add description property so that you can make comments on the quality (e.g. ``Definitive book
  on GR'').
\item Potentially use Wikipedia categories to categorize books in list and in directories.
\end{enumerate}

\section{Org Capture Extension}

Consider using \href{https://github.com/sprig/org-capture-extension}{org-capture-extension} which
allows you to run org-capture from a browser.

\section{Git Bug}

Consider using \href{https://github.com/MichaelMure/git-bug}{git-bug} instead of this file to track
issues and improvements.

\section{Eyebrowse}

Use \href{https://github.com/wasamasa/eyebrowse}{eyebrowse} to easily navigate between Emacs workstations.

\section{Setup BBDB or other contact list for GNUs}

bbdb keeps a contact list that can be used in gnus. Set this up, or look for an alternative
(potentially one that does not require manual additions). Note that there is also a company backend
for this, company-bbdb.

\section{Company specifies backend for each completion}

It would be nice to be able to show backend used for each completion with company. This could be a
configurable setting in company, for instance. Maybe look at company-box for a possible way to
implement this.

\section{Implement method to see what is causing lag in Emacs}

It would be really nice to be able to see what is causing Emacs to lag in certain contexts. This
could be enabled by calling a sort of record function and then disabling it when done. When the
record function is toggled off, display a list of functions called and how much collective time they
took, ordered by that time.

\section{Create a new gdb many windows mode that spans 2 screens}

This should also have a function to restore windows. It should delegate most responsibility to GDB\@.

\section{Set C modes indent automatically from clang-format}

Instead of automatically using a configured indent size and style (tabs or not) set it dynamically
when entering a c-mode buffer based one of the following in descending order of preference: (1) the
clang-format file used in that project, (2) the style already used and (3) your fallback style. This
Emacs manual
\href{https://www.gnu.org/software/emacs/manual/html_node/emacs/C-Indent.html#C-Indent}{link} should
be useful. Also, look at this
\href{https://www.reddit.com/r/emacs/comments/7uq9w1/replace_emacs_c_autoformatting_with_clangformat/dtmbq1j?utm_source=share&utm_medium=web2x}{reddit
  forum post}.

Consider whether it makes sense to do this in other programming modes as well. How do those modes
set the indentation style?

\section{Change company completion face}

The red text face used for company completion is ugly. Find a thematic alternative.

\section{Fix YCMD no flags error}

YCMD sometimes complains and repeatedly throws a flags missing error. Figure out what causes this
(look at the YCMD code), then figure out how to solve it. This probably involves reconfiguration the
ycm extra conf file.

\section{Keep company-box?}

Consider the functionality of company-box and potentially get rid of it.

\section{Completion drop-down vs fill-in}

Sometimes completions provide a drop-down and sometimes it doesn't and instead presents a sole
completion as a differently colored rest of the word. What logic regulates this? Is it when only a
single completion is available? Is this disadvantageous for a sole completion when the dropdown
could also provide documentation for the completion?

\section{Improve GDB behavior for opening windows}

The current GDB split window behavior isn't great. It splits up existing windows into overly small
sizes. I'm not exactly sure what I want here, but I'd like a strong priority given to the gdb buffer
and the source buffer. In any event these two links should be helpful:
\href{https://emacs.stackexchange.com/questions/38945/m-x-gdb-dont-create-new-frames}{this} and
\href{https://www.gnu.org/software/emacs/manual/html_node/elisp/Display-Action-Functions.html}{this}.

\end{document}
%%% Local Variables:
%%% mode: latex
%%% TeX-master: t
%%% End:
